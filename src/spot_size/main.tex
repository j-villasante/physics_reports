\documentclass[twocolumn]{article}
\usepackage[spanish]{babel}
\usepackage{caption}
\usepackage{graphicx}
\usepackage{amsmath}
\setlength{\parindent}{0pt}

\author{Josué Villasante}
\title{Medición del tamaño de spot de un laser}

\begin{document}
	\maketitle

	\section{Objetivo}
		El objetivo fue obtener el tamaño del spot del láser a partir del cambio de intensidad a medida el láser es bloqueado por el filo de una cuchilla perpendicular al haz.
	
	\section{Motivación}
		Al analizar la potencia de entrada y de salida del haz del láser a través de una fibra óptica se observó que aproximadamente solo el 30\% lograba atravesar. El lente del coalimador no es el correcto para el láser que tenemos. Por lo tanto, es necesario medir el tamaño del spot del láser a fin de poder conocer cuál es el lente correcto.
	
	\section{Procedimiento}
		El láser utilizado tuvo una longitud de onda de 633 nm y una potencia menor a 20 mW. Primero se colocó la cuchilla a aproximadamente unos 35 cm del láser y se bloqueo completamente el haz. El potenciometro se colocó a 50 cm del láser. Luego, por cada movimiento de 0.1 mm se registró la intensidad del láser en mW. Al iniciar, la base marcaba 5 mm.
		El mismo procedimiento de realizó colocando la cuchilla a un distancia de 1 m del laser y el potenciometro a 1.1 m.
\end{document}